
% Load configs
\documentclass{CV}

\begin{document}

% ------ HEADER ------
\begin{center}
{\textbf{\LARGE Mirko LEDDA}}\\
Sharon Aviran Lab\\
Biomedical Engineering \& Genome Center\\
University of California, Davis, USA\\
Phone: \texttt{(510) 717-4889}\\
Email: \mail{maledda@ucdavis.edu}\\
Webpage:\href{https://mirkoledda.github.io}{~https://mirkoledda.github.io}\\
\medskip
\medskip
\begin{minipage}[c]{0.92\textwidth}
My objective is to obtain a 3-month internship in Computational Biology as part of the requirements of the Designated Emphasis in Biotechnology graduate program. I am result oriented and thrives in multinational teams and interdisciplinary research environments.
\end{minipage}
\end{center}
\medskip

% ------ AREAS OF SPECIALIZATION ------
\section*{Areas of specialization}
\centerline{Genetics - Genomics - Computational Biology - Structural Biology - Molecular Biology}
\centerline{Bioinformatics - Statistics - Machine Learning - Software Development}
~

% ------ EDUCATION ------
% \edu{year}{degree}{institution}
\section*{Education}
\years{2014-} \textbf{Ph.D. in  Integrative Genetics and Genomics with Emphasis in Biotechnology (DEB*)}\\
University of California at Davis, CA, USA \hfill Expected graduation in Early 2019\\
~\\
\begin{tabular}{rl}
    Thesis title:&Hairpin in a haystack: structure-guided search for functional RNA elements\\
    Thesis advisor:& Prof. Sharon Aviran, Biomedical Engineering \& Genome Center, UC Davis\\
    DEB advisor:& Prof. Dr. Judith A. Kjelstrom, director, UC Davis Biotechnology Program\\
\end{tabular}\\
~\\
\begin{footnotesize}
*The \href{www.deb.ucdavis.edu}{DEB graduate program} is an inter-graduate group program involving faculty and students from 29 STEM disciplines. It is focused on providing cross-disciplinary training in critical areas of biomolecular research, promoting interdisciplinary team science, bioethics, entrepreneurship and professionalism as well as coordinating training in a biotechnology or life science company.
\end{footnotesize}\\
~\\
~
\years{2008} \textbf{B.Sc. in Life Sciences with Emphasis in Biotechnology}\\
University of Applied Sciences (HES-SO), Sion, Switzerland\\
~\\
\begin{tabular}{rl}
    Topic:& Strategies for the study of genes with unknown functions in \textit{Streptomyces}\\
    Advisors:& Prof. Sergio Schmid, Institute of Life Technologies, University of Applied \\
    &~~~~~~Sciences (HES-SO), Sion, Switzerland\\
    & Prof. Anna Maria Puglia, Dept. of Biological, Chemical and Pharmaceutical\\
    & ~~~~~~Sciences and Technologies, University of Palermo, Italy\\
\end{tabular}\\
~

% ------ SCIENTIFIC AND PROFESSIONAL EXPERIENCE ------
% \professional{year}{position}{institution}{description}
\section*{Professional experience}
\professional{2009-2014}{Research Assistant}{Nestl\'e Research Center, Lausanne, Switzerland}{Prof. Johannes le Coutre}{
Project and Lab manager for research on taste perception physiology. My contributions include:
\begin{itemize}[label=\raisebox{0.25ex}{\tiny$\bullet$},leftmargin=*]
\setlength\itemsep{0em}
\item[--] Genome-wide association studies (GWAS) of human metabolism and taste perception; Discovered new biomarkers for the health-status of the gastrointestinal tract and new genetic drivers of bitter taste perception.
\item[--] Statistical methods development to analyze human taste phenotypic data.
\item[--] Set up of the single cell Ca$^{2+}$-imaging technique and development of computational tools for automated data analysis.
\item[--] Functionalization and \emph{in vitro} validation of several human, feline and rat GPCRs/TRP channels in mammalian cells (HEK, Hela, CHO, Chem-1 and primary rat DRG neurons). Receptor-interaction studies using siRNAs and co-expression approaches.
\item[--] Method development for the expression and purification of water-insoluble proteins in \emph{E. coli}.
\end{itemize}~
\vspace{-12pt}
}
\professional{2008-2014}{Soldier, specialist in biological weapons}{Swiss Army (Labor Spiez), Spiez, Switzerland}{Dr. Christian Beuret}{Development and validation of laboratory techniques for the identification of pathogenic Bacteria, Viruses and Toxins.}

% ------ TEACHING ------
% \teaching{quarter + year}{position}{course name}{description}
\section*{Teaching experience}
\teaching{2018}{Guest Lecturer}{Quantitative Genetics and Selection Theory, UC Davis}{1h30 lecture on fundamental concepts in Machine Learning.}
\teaching{2017}{Lecturer}{Machine Learning Workshop for the Plant Sciences Dept., UC Davis}{4h lecture on fundamental concepts in Machine Learning for the Ross-Ibarra, Knapp and Runcie Labs.}
\teaching{}{Guest Lecturer}{Topics in Biomedical Engineering: Computational Genomics, UC Davis}{Two 2h lectures on fundamental concepts in Machine Learning.}
\teaching{2016}{Teaching assistant}{Quantitative Genetics and Selection Theory, UC Davis}{Teaching R programming and the mathematical bases of selection theory in lab sessions.}
\teaching{2015}{Course development}{Quantitative Genetics and Selection Theory, UC Davis}{Preparation of the teaching material for this newly proposed class.}

% ------ AWARDS ------
% \award{year}{name}{institution}{description}
\section*{Awards}
\award{2017}{Graduate Student Travel Award}{UC Davis}{Competitive award to cover the cost to attend, as a speaker, the \emph{2017 [BC]2 Basel Computational Biology Conference} in Basel, Switzerland.}
\award{2016}{Registration Bursary}{Wellcome Genome Campus Scientific Conferences}{Competitive award to cover the cost to attend, as a speaker, the \emph{2016 Computational RNA Biology Conference} in Cambridge, UK.}
\award{}{Summer Graduate Student Researcher Award}{UC Davis}{3-months support for graduate research in engineering, computer science, and disciplines with engineering-related applications and methods.}

% MEMBERSHIPS ----
% \member{year}{name}{description}
\section*{Membership}
\member{2018}{The RNA Society}{Student member}


% ------ SCIENCE OUTREACH ------
% \outreach{year}{event}{location}{description}
\section*{Community Service}
\outreach{2015-}{Graduate Student Association (GSA)}{UC Davis}{Representative for the IGG graduate program.}
\outreach{2017}{IGG Annual Colloquium}{UC Davis}{Member of the organizing committee.}
\outreach{}{Teen Biotech Challenge 2017}{DEB, UC Davis}{Judge for System and Computational Biology websites.}
\outreach{}{Topics in Biomedical Engineering: Computational Genomics}{UC Davis}{Mentored three students for their final projects.}
\outreach{2016}{Teen Biotech Challenge 2016}{DEB, UC Davis}{Judge for System and Computational Biology websites.}
\outreach{2015}{Science in the Siskiyous}{Dunsmuir High School, Dunsmuir, CA, USA}{Presented biology research and taught basic genetic concepts to three $9^{\text{th}}\text{~to~}12^{\text{th}}$ grade high-school classes.}
\outreach{}{Science vs Fiction}{Senior Center, Davis, CA, USA}{Presented common scientific misconceptions followed by an open discussion with seniors.}
\outreach{}{IGG program}{UC Davis}{Mentor for all incoming international IGG students and mentor for a $1^{\text{st}}$ year IGG student.}

% ------ PUBLICATIONS ------
% \pubArticle{year}{authors}{title}{journal}{issue}{OPTIONAL doi}
\section*{Publications \footnotesize\text{( * indicates co-authorship)}}\medskip
% Published articles ------
\pubArticle{2018}{\underline{Ledda M.} and Aviran S.}{patteRNA: transcriptome-wide search for functional RNA elements via structural data signatures}{Genome Biology}{in press}
\pubArticle{2016}{Choudhary K., Shih N.P., Deng F., \underline{Ledda M.}, Li B. and Aviran S.}{Metrics for rapid quality control in RNA structure probing experiments}{Bioinformatics}{32(23): 2575-3583}{10.1093/bioinformatics/btw501}
\pubArticle{}{Deng F.*, \underline{Ledda M.*}, Vaziri S. and Aviran S.}{Data-directed RNA secondary structure prediction using probabilistic modeling}{RNA}{22(8): 1109-19}{10.1261/rna.055756.115}
\pubArticle{}{Michlig Gonz�lez S., Meylan Merlini J., Beaumont M., \underline{Ledda M.}, Tavenard A., Mukherjee R., Camacho S and le Coutre J.}{Acute Effects of single ingestion of TRPV1, TRPA1 and TRPM8 agonists on the energetic metabolism and the autonomic activity in healthy subjects}{Scientific Reports}{6: 20795}{10.1038/srep20795}
\pubArticle{2014}{Rueedi R.*, \underline{Ledda M.*}, Nicholls A.W., Salek R.M., Marques-Vidal P., Morya E., Sameshima K., Montoliu I., Da Silva L., Collino S. et al.}{Genome-wide association study of metabolic traits reveals novel gene-metabolite-disease links}{PLoS Genetics}{10(2)}{10.1371/journal.pgen.1004132}
\pubArticle{2013}{\underline{Ledda M.*}, Kutalik Z.*, Destito M.C.S., Souza M.M., Cirillo C. a., Zamboni A., Martin N., Morya E., Sameshima K., Beckmann J.S. et al.}{GWAS of human bitter taste perception identifies new loci and reveals additional complexity of bitter taste genetics}{Human Molecular Genetics}{23: 259-267}{10.1093/hmg/ddt404}
\pubArticle{}{Godinot N., Yasumatsu K., Barcos M.E., Pineau N., \underline{Ledda M.}, Viton F., Ninomiya Y., le Coutre J. and Damak S.}{Activation of tongue-expressed GPR40 and GPR120 by non caloric agonists is not sufficient to drive preference in mice}{Neuroscience}{250: 20-30}{10.1016/j.neuroscience.2013.06.043}
\pubArticle{}{Montoliu I.*, Genick U.*, \underline{Ledda M.}, Collino S., Martin F.P., Le Coutre J. and Rezzi S.}{Current status on genome-metabolome-wide associations: An opportunity in nutrition research}{Genes and Nutrition}{8: 19-27}{10.1007/s12263-012-0313-7}
\pubArticle{2011}{Genick U.K., Kutalik Z., \underline{Ledda M.}, Souza Destito M.C., Souza M.M., Cirillo C. a., Godinot N., Martin N., Morya E., Sameshima K. et al.}{Sensitivity of genome-wide-association signals to phenotyping strategy: The PROP-TAS2R38 taste association as a benchmark}{PLoS One}{6(11)}{10.1371/journal.pone.0027745}

% Submitted / In prep articles ------
% \pubRev{year}{authors}{title}{journal}
% \pubSub{year}{authors}{title}{journal}
% \pubPrep{year}{authors}{title}{journal}

% \subsection*{Manuscripts submitted / in-preparation}

% ------ PATENTS ------
% \pubPatent{year filed}{inventors}{title}{patent office}{patent #}{year issued}
\section*{Patents}
\pubPatent{2014}{Genick U.K., \underline{Ledda M.}, Montoliu I., Le Coutre J., Rezzi S., Collino S., Martin F.P., Da Silva L.}{Genetic and urine-derived markers of human metabolic and gut microbial states}{US Patent Office}{US Patent 20,150,160,191}{2015}
\pubPatent{2012}{Genick U.K., \underline{Ledda M.}, Montoliu I., Le Coutre J., Rezzi S., Collino S., Martin F.P., Da Silva L.}{Genetic and urine-derived markers of human metabolic and gut microbial states}{European Patent Office}{EP2687845 A1}{2014}

% ------ TALKS AND POSTERS ------
% \pubTalk{year}{authors}{title}{conference}{location}{date}{description}
\section*{Talks and Posters}
\pubTalk{2017}{\underline{Ledda M.} and Aviran S.}{patteRNA: Transcriptome-wide search for functional RNA elements via structural data signatures}{[BC]2 Basel Computational Biology Conference}{Congress Center, Basel, Switzerland}{September 13-$15^\text{th}$}{Speaker - 20min talk}
\pubTalk{}{\underline{Ledda M.} and Aviran S.}{Transcriptome-wide search for functional RNA elements via structural data signatures}{Genome Research Day}{23andMe, Mountain View, CA}{June 1st}{Poster}
\pubTalk{2016}{\underline{Ledda M.}, Deng F., Vaziri S., and Aviran S.}{Data-directed RNA secondary structure prediction using probabilistic modeling}{Computational RNA Biology Conference}{Wellcome Trust, Cambridge, UK}{October 17-$19^\text{th}$}{Speaker - 15min talk}

% ------ LANGUAGES ------
\section*{Hobbies/Interests}
\centerline{Soccer - Alpine Ski - Hiking - Taking (many) pictures - Building servers at home}
~

\begin{center}
\Large \textbf{References upon request}
\end{center}

% ------ LANGUAGES ------
%\section*{Languages}
%\begin{multicols}{3}
%\begin{itemize}[label=\raisebox{0.25ex}{\tiny$\bullet$},leftmargin=*]
%\item French: Mother Tongue
%\item Italian: Mother Tongue
%\item English: Fluent
%\end{itemize}
%\end{multicols}
%~

% ------ COMPUTER SKILLS ------
%\section*{Programming Languages}
%\centerline{Python - R - Matlab - Perl - Bourne Shell}
%~
%\section*{Computer skills}
%\emph{Programming} \tab{Python - R - Matlab - Perl - Bourne Shell}\\
%\emph{Typesetting} \tab{LaTex - Markdown - Microsoft Office}\\

\end{document}